\chapter*{Conclusion}
\label{sec:conclusion}

% Présentation générale du micro-gyroscope
Dans cet article a été tout d'abord présenté le fonctionnement mécanique des micro-gyroscopes inertiels capacitifs, le fonctionnement de la détection et de l'excitation des modes de résonances ainsi que les différents éléments de l'électronique de contrôle et de lecture.\\

% Différentes non-idéalités et problématiques
Les non-idéalités présentes dans le système ont été introduites. Cela a permis de dégager les problématiques sur le contrôle de ces micro-capteurs.\\

% Exposé des différentes méthodes de contrôle pour répondre aux problématiques
Finalement, de multiples recherches actuelles sur les méthodes de contrôle permettant de répondre à ces problématiques ont été présentées.\\

% Ouverture
Malgré la multiplicité des méthodes de contrôle appliquées au micro-gyroscope, aucune méthode ne s'est réellement dégagée des autres pour devenir le standard. De plus, de nombreuses problématiques n'ont été que survolées dans cet article. Parmi elles la minimisation de l'erreur de quadrature, la prise en compte des différents bruits présents dans le système, les comportements du micro-gyroscope à court et à long terme, et notamment la stabilité du biais sur de longues périodes.

Les recherches sur les méthodes de contrôle des micro-gyroscopes vont ainsi se poursuivre pour encore en améliorer les performances.
